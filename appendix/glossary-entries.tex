% Acronyms
%\newacronym[description={\glslink{apig}{Application Program Interface}}]
%    {api}{API}{Application Program Interface}
%
%\newacronym[description={\glslink{umlg}{Unified Modeling Language}}]
%    {uml}{UML}{Unified Modeling Language}

\newacronym[description={\glslink{ide}{Integrated Development Environment}}]
    {ide}{IDE}{Integrated Development Environment}

\newacronym[description={\glslink{uig}{User Interface}}]
    {ui}{UI}{User Interface}

\newacronym[description={\glslink{uxg}{User Experience}}]
    {ux}{UX}{User Experience}

\newacronym[description={\glslink{wcagg}{Web Content Accessibility Guidelines}}]
    {wcag}{WCAG}{Web Content Accessibility Guidelines}

% Glossary entries
%\newglossaryentry{apig} {
%    name=\glslink{api}{API},
%    text=Application Program Interface,
%    sort=api,
%    description={in informatica con il termine \emph{Application Programming Interface API} (ing. interfaccia di programmazione di un'applicazione) si indica ogni insieme di procedure disponibili al programmatore, di solito raggruppate a formare un set di strumenti specifici per l'espletamento di un determinato compito all'interno di un certo programma. La finalità è ottenere un'astrazione, di solito tra l'hardware e il programmatore o tra software a basso e quello ad alto livello semplificando così il lavoro di programmazione}
%}
%
%\newglossaryentry{umlg} {
%    name=\glslink{uml}{UML},
%    text=UML,
%    sort=uml,
%    description={in ingegneria del software \emph{UML, Unified Modeling Language} (ing. linguaggio di modellazione unificato) è un linguaggio di modellazione e specifica basato sul paradigma object-oriented. L'\emph{UML} svolge un'importantissima funzione di ``lingua franca'' nella comunità della progettazione e programmazione a oggetti. Gran parte della letteratura di settore usa tale linguaggio per descrivere soluzioni analitiche e progettuali in modo sintetico e comprensibile a un vasto pubblico}
%}

% i termini che ho inserito con Gabriel -----------------------------------------------------

\newglossaryentry{buildg} {
    name=Build,
    text=build,
    sort=bld,
    description={indica la trasformazione del codice in un prodotto software eseguibile}
}

\newglossaryentry{cassacomuneg} {
    name=Cassa Comune,
    text=cassa comune,
    sort=cas,
    description={viene utilizzato questo termine per indicare i fondi dati dagli operatori aziendali per coprire i pasti}
}

\newglossaryentry{componentig} {
    name=Componenti,
    text=componenti,
    sort=comp,
    description={sono un insieme di \emph{widget} e di elementi che insieme costituiscono un prodotto software}
}

\newglossaryentry{dartg} {
    name=Dart,
    text=Dart,
    sort=dart,
    description={linguaggio di programmazione \textit{open-source} sviluppato da Google. 
    È il linguaggio principale utilizzato per scrivere applicazioni con \emph{\gls{flutterg}}. Dart è noto per la sua velocità ed efficienza nella creazione di applicazioni mobili e web.
    Risulta inoltre staticamente tipizzato, cioè consente una dichiarazione esplicita dei tipi delle variabili e garantisce maggiore robustezza in programmazione}
}

\newglossaryentry{firebaseg} {
    name=Firebase,
    text=Firebase,
    sort=fire,
    description={piattaforma di sviluppo di app mobile di Google che offre una serie di servizi tra cui \emph{database} in tempo reale, autenticazione utente, \emph{hosting} di applicazioni e molto altro. 
    È ampiamente utilizzato per la costruzione di app mobile e web in modo rapido e scalabile, grazie alle funzionalità \emph{cloud}, di notifica e di monitoraggio in \emph{real time}}
}

\newglossaryentry{flutterg} {
    name=Flutter,
    text=Flutter,
    sort=flutt,
    description={\textit{framework} \textit{open-source} di Google per lo sviluppo di applicazioni mobile, desktop e webapp utilizzando il linguaggio \emph{\gls{dartg}}. È basato su \emph{widget} personalizzabili, puntando su un rapido sviluppo, eccellenti performance, una comunità attiva e supporto per molte piattaforme}
}

\newglossaryentry{ideg} {
    name=\glslink{ide}{IDE},
    text=IDE,
    sort=ide,
    description={è un ambiente di sviluppo integrato che supporta i programmatori nello sviluppo e nel \emph{debug} del codice}
}

\newglossaryentry{quotastornatag} {
    name=Quota Stornata,
    text=quota stornata,
    sort=storn,
    description={indica i soldi che il singolo utente deve dare o ricevere dagli altri utenti per i pasti effettuati e le spese sostenute}
}

\newglossaryentry{uig} {
    name=\glslink{ui}{UI},
    text=UI,
    sort=ui,
    description={indica l'interfaccia grafica che viene utilizzata per le comunicazioni tra uomo e macchina}
}

\newglossaryentry{uxg} {
    name=\glslink{ux}{UX},
    text=UX,
    sort=ux,
    description={indica l'insieme di sensazioni e ricordi che una persona prova quando si rapporta con un prodotto, cioè tutti gli aspetti che condizionano il prodotto per consentire all'utente di utilizzarlo e capirlo con facilità}
}

\newglossaryentry{wcagg} {
    name=\glslink{wcag}{WCAG},
    text=WCAG,
    sort=wcag,
    description={si tratta di una serie di linee guida per l'accessibilità, fornisce una serie di criteri tecnici per rendere siti web, applicazioni e altri contenuti facilmente utilizzabili da tutti i tipi di utente}
}
