\chapter{Introduzione}
\label{cap:introduzione}

%Introduzione al contesto applicativo.\\
%
%\noindent Esempio di utilizzo di un termine nel glossario \\
%\gls{api}. \\
%
%\noindent Esempio di citazione in linea \\
%\cite{site:agile-manifesto}. \\
%
%\noindent Esempio di citazione nel pie' di pagina \\
%citazione\footcite{womak:lean-thinking} \\

\section{L'azienda}

RiskApp S.r.l. (Figura \ref{fig:riskapp}) è un'azienda con sede a Conselve (PD) che si occupa di sviluppo software per il mondo assicurativo.\newline
È stata fondata nel 2016 e il suo \textit{core business} è lo sviluppo e il mantenimento dell'omonima applicazione, che viene costantemente aggiornata ed estesa per garantire un prodotto che possa rispondere ad ogni esigenza.\newline
Il principale punto di forza di questa piattaforma è quello di stimare le possibili perdite economiche di un’impresa attraverso un algoritmo proprietario che, anche attraverso l’uso dell'intelligenza artificiale, valuta il rischio raccogliendo e combinando una moltitudine di dati da diverse fonti.\newline
Il personale aziendale lavora costantemente per migliorare i propri servizi, ragionando sui possibili problemi che l'utente e l'aziende possono andare incontro, fanno riunioni e call per capire come migliorare e ampliare la piattaforma, tutto svolto in un clima di calma e rispetto tra colleghi.

\begin{figure}[!h] 
    \centering 
    \includegraphics[width=0.4\columnwidth]{RiskAPP} 
    \caption{Logo dell'azienda RiskApp}\label{fig:riskapp}
\end{figure}

\section{L'idea}

Per poter gestire le spese per i pasti aziendali, è stato scelto di sviluppare un'app mobile che permetta di monitorare i versamenti degli utenti, scegliere il piatto del giorno da un menu condiviso e monitorare la {\gls{cassag}}\glsfirstoccur.\newline
Deve essere gestita l'autenticazione di ogni utente, dividendo tra utente semplice e utente amministratore e permettere il controllo delle presenze in azienda durante i pranzi. \newline
Ogni utente potrà aggiungere un piatto nel menu, proporre il pasto del giorno, monitorare la sua {\gls{storng}}\glsfirstoccur e la cassa comune, indicare le spese effettuate e modificare i dati personali. \newline
L'amministratore potrà anche gestire le presenze e le spese effettuate dagli stagisti.
L'applicazione dovrà essere sviluppata con {\gls{fluttg}}\glsfirstoccur, {\gls{dartg}}\glsfirstoccur e {\gls{fireg}}\glsfirstoccur.

%da modificare - vedere alla fine ----------------------------------------------------------
\section{Organizzazione del testo}

\begin{description}
    \item[{\hyperref[cap:processi-metodologie]{Il secondo capitolo}}] descrive ...
    
    \item[{\hyperref[cap:descrizione-stage]{Il terzo capitolo}}] approfondisce ...
    
    \item[{\hyperref[cap:analisi-requisiti]{Il quarto capitolo}}] approfondisce ...
    
    \item[{\hyperref[cap:progettazione-codifica]{Il quinto capitolo}}] approfondisce ...
    
    \item[{\hyperref[cap:verifica-validazione]{Il sesto capitolo}}] approfondisce ...
    
    \item[{\hyperref[cap:conclusioni]{Nel settimo capitolo}}] descrive ...
\end{description}


%da tenere e non toccare --------------------------------------------------------------

Riguardo la stesura del testo, relativamente al documento sono state adottate le seguenti convenzioni tipografiche:
\begin{itemize}
	\item gli acronimi, le abbreviazioni e i termini ambigui o di uso non comune menzionati vengono definiti nel glossario, situato alla fine del presente documento;
	\item per la prima occorrenza dei termini riportati nel glossario viene utilizzata la seguente nomenclatura: \emph{parola}\glsfirstoccur;
	\item i termini in lingua straniera o facenti parti del gergo tecnico sono evidenziati con il carattere \emph{corsivo}.
\end{itemize}
