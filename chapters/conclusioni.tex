\chapter{Conclusioni}
\label{cap:conclusioni}

\intro{In conclusione a tutto il percorso di stage, questo capitolo riporta un resoconto degli obiettivi e dei requisiti raggiunti, seguito poi da un'analisi personale del progetto svolto.}\\ 

\section{Raggiungimento degli obiettivi}
In riferimento alle notazioni indicate nel \hyperref[sec:obbiettivi]{primo capitolo} e nel \hyperref[sec:requisiti]{terzo capitolo} e in riferimento alle Tabelle \ref{tab:obiettivi-richiesti}, \ref{tab:requisiti-funzionaliuno}, \ref{tab:requisiti-funzionalidue}, \ref{tab:requisiti-qualitativi} e \ref{tab:requisiti-vincolo}, di seguito sono riportati gli obbiettivi e i requisiti con il loro stato di completamento.\newline
Alcuni obbiettivi e alcuni requisiti non sono stati soddisfatti perchè è stato deciso di impiegare più tempo per il raggiungimento degli obiettivi obbligatori, dedicando più tempo allo studio e cercando di migliorare alcune funzioni dell'applicazione rischiesta.\newline

\subsection{Obiettivi raggiunti}
%Si farà riferimento ai requisiti secondo le seguenti notazioni:
%\begin{itemize}
%    \item O per i requisiti obbligatori, vincolanti in quanto obiettivo primario richiesto dal committente;
%    \item D per i requisiti desiderabili, non vincolanti o strettamente necessari, ma dal riconoscibile valore aggiunto;
%    \item F per i requisiti facoltativi, rappresentanti valore aggiunto non strettamente competitivo.
%\end{itemize}
%Le sigle precedentemente indicate saranno seguite da una coppia sequenziale di numeri, identificativo del requisito.

\begin{table}[htb]%
    \caption{Tabella degli obiettivi raggiunti}
    \label{tab:obiettivi-raggiunti}
    \begin{tabularx}{\textwidth}{lXl}
    \hline
    \textbf{Obiettivo} & \textbf{Descrizione} & \textbf{Stato}\\
    \hline\hline
    O01     & accesso tramite credenziali & Soddisfatto \\
    \hline
    O02     & pannello di controllo degli utenti; & Soddisfatto \\
    \hline
    O03     & modifica o aggiunta delle spese & Soddisfatto \\
    \hline
    O04     & monitoraggio della cassa comune & Soddisfatto \\
    \hline
    O05     & controllo presenza in azienda di una persona durante i pranzi & Soddisfatto \\
    \hline
    O06     & impostare il menu del giorno & Soddisfatto \\
    \hline
    O07     & scelta di un piatto dal menu & Soddisfatto \\
    \hline
    D01     & integrazione di chatGPT per consigliare alcune ricette & Non soddisfatto \\
    \hline
    F01     & test a livello di frontend & Non soddisfatto \\
    \hline
    \end{tabularx}
\end{table}%

\newpage

\subsection{Requisiti funzionali raggiunti}

\begin{table}[htb]%
\caption{Tabella del tracciamento dei requisti funzionali dall'1 al 17 raggiunti}
\label{tab:reqfunzionali-raggiunti}
\begin{tabularx}{\textwidth}{lXl}
\hline
\textbf{Requisito} & \textbf{Descrizione} & \textbf{Stato}\\
\hline\hline
RFO-1     & L'utente effettua l'accesso all'app inserendo la propria password e la propria mail & Soddisfatto \\
\hline
RFO-2     & L'utente si registra nel database inserendo il proprio nome, cognome, mail e password & Soddisfatto\\
\hline
RFO-3     & Si visualizza la \emph{\gls{cassacomuneg}} salvata nel database nell'app & Soddisfatto \\
\hline
RFO-4     & L'utente visualizza la propria \emph{\gls{quotastornatag}} salvata nel database nell'app & Soddisfatto \\
\hline
RFO-5     & Si visualizza la lista dei piatti proposti del giorno nell'app & Soddisfatto \\
\hline
RFO-6     & Si visualizza la lista delle transazioni nell'app & Soddisfatto \\
\hline
RFO-7     & L'utente aggiunge una nuova transazione nell'app, indicando i soldi e la data e salva la transazione nel database & Soddisfatto \\
\hline
RFO-8     & L'utente amministratore aggiunge la spesa effettuata da uno stagista nell'app, indicando la data e quanto ha speso e lo salva nel database & Soddisfatto \\
\hline
RFO-9     & L'utente indica la spesa che ha effettuato nell'app, riportando i soldi e la data e lo salva nel database & Soddisfatto \\
\hline
RFO-10    & L'utente indica nell'app i soldi che ha inviato a un altro utente registrato nel database e salva la transazione nel database & Soddisfatto \\
\hline
RFO-11    & L'utente elimina una transazione presente nel database dall'app & Soddisfatto \\
\hline
RFO-12    & Si visualizza il menu che contiene la lista dei piatti dall'app & Soddisfatto \\
\hline
RFO-13    & L'utente aggiunge un nuovo piatto nell'app, indicando il nome del piatto, gli ingredienti e la ricetta e lo salva nel database & Soddisfatto \\
\hline
RFO-14    & L'utente elimina un piatto presente nel database dall'app & Soddisfatto \\
\hline
RFO-15    & L'utente propone un piatto da mangiare a pranzo selezionandolo dal menu & Soddisfatto \\
\hline
RFO-16    & L'amministratore visualizza la \emph{\gls{quotapastog}} dall'app & Soddisfatto \\
\hline
RFO-17    & L'amministratore modifica la \emph{\gls{quotapastog}} dall'app e salva il nuovo valore nel database & Soddisfatto \\
\hline
\end{tabularx}
\end{table}%

\begin{table}%
\caption{Tabella del tracciamento dei requisti funzionali dal 18 al 35 raggiunti}
%\label{tab:requisiti-funzionali}
\begin{tabularx}{\textwidth}{lXl}
\hline
\textbf{Requisito} & \textbf{Descrizione} & \textbf{Stato}\\
\hline\hline
RFO-18    & L'amministratore visualizza la \emph{\gls{quotastornatag}} degli stagisti dall'app & Soddisfatto \\
\hline
RFO-19    & L'amministratore visualizza la lista con indicato i giorni di presenza degli stagisti dall'app & Soddisfatto \\
\hline
RFO-20    & L'amministratore modifica la lista con indicato i giorni di presenza degli stagisti dall'app e salva le modifiche nel database & Soddisfatto \\
\hline
RFO-21    & L'utente visualizza la lista con indicati i propri giorni di presenza a pranzo dall'app & Soddisfatto \\
\hline
RFO-22    & L'utente modifica la lista con indicati i propri giorni di presenza a pranzo dall'app e salva le modifiche nel database & Soddisfatto \\
\hline
RFO-23    & L'utente si disconnette dall'app & Soddisfatto \\
\hline
RFO-24    & L'utente visualizza i propri dati dall'app & Soddisfatto \\
\hline
RFO-25    & L'utente visualizza la propria mail dall'app & Soddisfatto \\
\hline
RFO-26    & L'utente visualizza il proprio nome dall'app & Soddisfatto \\
\hline
RFO-27    & L'utente visualizza il proprio cognome dall'app & Soddisfatto \\
\hline
RFO-28    & L'utente visualizza il proprio UID Satispay dall'app & Soddisfatto \\
\hline
RFO-29    & L'utente modifica i propri dati dall'app e salva le modifiche nel database & Soddisfatto \\
\hline
RFO-30    & L'utente modifica la propria password dall'app e salva la nuova password nel database & Soddisfatto \\
\hline
RFO-31    & L'utente modifica il proprio nome dall'app e salva il nuovo nome nel database & Soddisfatto \\
\hline
RFO-32    & L'utente modifica il proprio cognome dall'app e salva il nuovo cognome nel database & Soddisfatto \\
\hline
RFO-33    & L'utente modifica il proprio UID Satispay e salva il nuovo UID nel database & Soddisfatto \\
\hline
RFD-34    & Viene chiesto a ChatGPT una possibile ricetta da proporre a pranzo & Non soddisfatto \\
\hline
RFD-35    & Si aggiunge la ricetta proposta da ChatGPT nel menu e si salva la ricetta nel database & Non soddisfatto \\
\hline
\end{tabularx}
\end{table}%

\newpage

\subsection{Requisiti qualitativi raggiunti}

\begin{table}[htb]%
\caption{Tabella del tracciamento dei requisiti qualitativi raggiunti}
\label{tab:reqqualitativi-raggiunti}
\begin{tabularx}{\textwidth}{lXl}
\hline
\textbf{Requisito} & \textbf{Descrizione} & \textbf{Stato}\\
\hline\hline
RQN-1    & Il codice \emph{front-end} deve essere coperto da test di unità & Non soddisfatto \\
\hline
\end{tabularx}
\end{table}%

\subsection{Requisiti di vincolo raggiunti}

\begin{table}[htb]%
\caption{Tabella del tracciamento dei requisiti di vincolo raggiunti}
\label{tab:reqvincolo-raggiunti}
\begin{tabularx}{\textwidth}{lXl}
\hline
\textbf{Requisito} & \textbf{Descrizione} & \textbf{Stato}\\
\hline\hline
RVO-1    & L'applicazione deve essere sviluppato con il \emph{framework} Flutter & Soddisfatto \\
\hline
RVO-2    & L'applicazione deve essere sviluppato con la piattaforma Firebase & Soddisfatto \\
\hline
RVO-3    & L'applicazione deve essere accessibile su cellulari con sistema operativo Android e iOS & Soddisfatto \\
\hline
RVO-4    & La mail che l'utente deve utilizzare per registrarsi nel database e accedere all'app deve essere fornita da RiskAPP & Soddisfatto \\
\hline
RVO-5    & La mail dell'utente non deve essere modificabile tramite app & Soddisfatto \\
\hline
\end{tabularx}
\end{table}%

%\section{Conoscenze acquisite}

\section{Valutazione personale}
Considerando che ho affrontato questo progetto, partendo senza avere conoscenze di Flutter o di Firebase, posso dire di essere soddisfatta del lavoro svolto, perchè ho imparato e studiato qualcosa di nuovo.\newline
Non nego che ci sono state diverse difficoltà, in particolare con Firebase, perchè non riuscivo a capire come collegare la console Firestore e lavorare sul database tramite codice Dart, ma con calma, pazienza e diversi test e ricerche, sono riuscita a capire come implementare le funzioni utili per permettere un corretto funzionamento dell'app, rendendola usabile per tutti i dispositivi.\newline
Ho imparato Flutter, Firebase, ho capito quanto sia utile progettare il modello di un software tramite Figma, perchè avere una idea visiva del progetto da creare, può aiutare a capire e a far capire su cosa si lavora e i relativi problemi da affrontare.\newline
Devo ringraziare il team di RiskApp, perchè sono stati pazienti e mi aiutavano quando lo chiedevo, permettendomi anche di lavorare in un ambiente tranquillo; se avessi lavorato in un ambiente diverso, probabilmente avrei trovato qualche difficoltà nello studio e nella creazione dell'app, perchè il clima che si creava in ufficio era un clima calmo, mantenendo il silenzio per consentire a tutti di lavorare e parlando ogni tanto solo per fare qualche commento o per confrontarsi sul lavoro da svolgere.\newline
