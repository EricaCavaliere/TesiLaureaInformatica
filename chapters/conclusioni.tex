\chapter{Conclusioni}
\label{cap:conclusioni}

\intro{In conclusione a tutto il percorso di stage, in questo capitolo viene fatto un resoconto di tutti gli obbiettivi raggiunti, riportando anche una valutazione personale di tutto il lavoro svolto, analizzando difficoltà successi e possibili punti da migliorare in futuro nel progetto.}\\

%\intro{Durante il mio lavoro, anche un altro stagista ha lavorato ad un'applicazione per gestire i pranzi in azienda, ma con qualche differenza,}\\

%\intro{In questo capitolo, si riporta un resoconto degli obiettivi e dei requisiti raggiunti, seguito poi da un'analisi personale del progetto svolto.}\\ 

\section{Valutazione personale}

Il progetto che ho sviluppato tratta di un prototipo di una applicazione per gestire i pranzi aziendali.\newline
Ho lavorato in solitaria, ma con il sostegno del mio tutor che è sempre stato disponibile quando chiedevo aiuto o consigli.\newline
\newline
Durante la mia permanenza in azienda, ci sono stati altri due stagisti che hanno lavorato al mio stesso progetto, ma concentrandosi su funzioni diverse.\newline
Nel mio caso, dovevo creare una gestione automatizzata per permettere il controllo della presenza a pranzo, un secondo stagista ha lavorato con ChatGPT per poter creare e eventualmente aggiungere dei piatti nel menu, mentre l'ultimo ha integrato Satispay all'app, per poter così gestire meglio le transazioni.\newline
\newline
Il motivo di questa distinzione  è per creare delle piccole bozze di una applicazione importante, capendo se le funzioni sviluppate possono essere utili per i dipendenti dell'azienda o se sono da scartare e quindi da non implementare nell'applicazione finale.\newline
\newline
Nonostante si tratta di una bozza, sono lo stesso contenta di aver affortontato questo progetto, perchè mi ha permesso di conoscere il linguaggio Dart con Flutter e il funzionamento di Firebase.\newline
Le prime due settimane sono state spese sullo studio delle tecnologie, poi ho dedicato un paio di settimene alla progettazione su Figma e poi ho sempre lavorato sulla codifica dell'app.\newline
\newline
Essendo un linguaggio nuovo per me, durante la fase di codifica ho avuto un po' di difficoltà, in particolare con Firebase, perchè non riuscivo a capire come collegare la console Firestore e lavorare sul database tramite codice Dart; ho risolto dopo una settimana di ricerche e di prove, anche con l'aiuto del mio tutor aziendale.\newline
\newline
Valutando tutto il lavoro svolto, ci sono alcune cose dell'app che cambierei, un esempio è la pagina transazione, andrei a gestire in modo diverso la visualizzazione dei dati nella lista, togliendo il messaggio di spesa o di invio soldi.\newline
Anche il \hyperref[sec:calendario]{calendario} modificherei, nonostante risponda alle esigenze dell'utente, se viene utilizzato tutti i giorni e se in un futuro si decide di fare delle modifche in uno o più giornate passate, questo può risultare pesante perchè, per come ho progettato il lavoro, si può modificare solo una data alla volta, facendo risultare all'utente un lavoro noioso e macchinoso.\newline
È stata una sfida anche implementare il \hyperref[sec:qrcode]{QrCode}, anche se non sono riuscita a implementare la funzione desiderata, sono lo stesso contenta del risultato ottenuto.\newline
%\newline
%Ci sono alcune funzioni o elementi che invece non cambierei, come la schermata Impostazioni perchè, anche se si tratta di una schermatata semplice, è stata una delle prime finestre che sono riuscita a creare senza tanti problemi e per l'utente finale risulta facile capire le funzioni presenti.\newline
\newline
Nonostante ci siano ancora molti punti su cui lavorare, sono riuscita a rispettare gli obiettivi obbligatori, sviluppando le funzioni principali richieste.\newline

\section{Raggiungimento degli obiettivi}

In riferimento alle notazioni indicate nel \hyperref[sec:obbiettivi]{primo capitolo} e nel \hyperref[sec:requisiti]{terzo capitolo} e in riferimento alle Tabelle \ref{tab:obiettivi-richiesti}, \ref{tab:requisiti-funzionaliuno}, \ref{tab:requisiti-funzionalidue}, \ref{tab:requisiti-qualitativi} e \ref{tab:requisiti-vincolo}, di seguito sono riportati gli obbiettivi e i requisiti con il loro stato di completamento.\newline
Alcuni obbiettivi e alcuni requisiti non sono stati soddisfatti perchè è stato deciso di impiegare più tempo per il raggiungimento degli obiettivi obbligatori, dedicando più tempo allo studio e cercando di migliorare alcune funzioni dell'applicazione richiesta.\newline

\begin{table}[htb]%
    \caption{Tabella degli obiettivi raggiunti}
    \label{tab:obiettivi-raggiunti}
    \begin{tabularx}{\textwidth}{lXl}
    \hline
    \textbf{Obiettivo} & \textbf{Descrizione} & \textbf{Stato}\\
    \hline\hline
    O01     & Accesso tramite credenziali & Soddisfatto \\
    \hline
    O02     & Pannello di controllo degli utenti & Soddisfatto \\
    \hline
    O03     & Modifica o aggiunta delle spese & Soddisfatto \\
    \hline
    O04     & Monitoraggio della cassa comune & Soddisfatto \\
    \hline
    O05     & Controllo presenza in azienda di una persona durante i pranzi & Soddisfatto \\
    \hline
    O06     & Impostare il menu del giorno & Soddisfatto \\
    \hline
    O07     & Scelta di un piatto dal menu & Soddisfatto \\
    \hline
    D01     & Integrazione di ChatGPT per consigliare alcune ricette & Non soddisfatto \\
    \hline
    F01     & Test a livello di frontend & Non soddisfatto \\
    \hline
    \end{tabularx}
\end{table}%

\newpage

%\subsection{Requisiti funzionali raggiunti}

\begin{table}[htb]%
\caption{Tabella del tracciamento dei requisiti funzionali dall'1 al 17 raggiunti}
\label{tab:reqfunzionali-raggiunti}
\begin{tabularx}{\textwidth}{lXl}
\hline
\textbf{Requisito} & \textbf{Descrizione} & \textbf{Stato}\\
\hline\hline
RFO-1     & L'utente effettua l'accesso all'app inserendo la propria password e la propria mail & Soddisfatto \\
\hline
RFO-2     & L'utente si registra nel database inserendo il proprio nome, cognome, mail e password & Soddisfatto\\
\hline
RFO-3     & Si visualizza la \emph{\gls{cassacomuneg}} salvata nel database nell'app & Soddisfatto \\
\hline
RFO-4     & L'utente visualizza la propria \emph{\gls{quotastornatag}} salvata nel database nell'app & Soddisfatto \\
\hline
RFO-5     & Si visualizza la lista dei piatti proposti del giorno nell'app & Soddisfatto \\
\hline
RFO-6     & Si visualizza la lista delle transazioni nell'app & Soddisfatto \\
\hline
RFO-7     & L'utente aggiunge una nuova transazione nell'app, indicando i soldi e la data e salva la transazione nel database & Soddisfatto \\
\hline
RFO-8     & L'utente amministratore aggiunge la spesa effettuata da uno stagista nell'app, indicando la data e quanto ha speso e lo salva nel database & Soddisfatto \\
\hline
RFO-9     & L'utente indica la spesa che ha effettuato nell'app, riportando i soldi e la data e lo salva nel database & Soddisfatto \\
\hline
RFO-10    & L'utente indica nell'app i soldi che ha inviato a un altro utente registrato nel database e salva la transazione nel database & Soddisfatto \\
\hline
RFO-11    & L'utente elimina una transazione presente nel database dall'app & Soddisfatto \\
\hline
RFO-12    & Si visualizza il menu che contiene la lista dei piatti dall'app & Soddisfatto \\
\hline
RFO-13    & L'utente aggiunge un nuovo piatto nell'app, indicando il nome del piatto, gli ingredienti e la ricetta e lo salva nel database & Soddisfatto \\
\hline
RFO-14    & L'utente elimina un piatto presente nel database dall'app & Soddisfatto \\
\hline
RFO-15    & L'utente propone un piatto da mangiare a pranzo selezionandolo dal menu & Soddisfatto \\
\hline
RFO-16    & L'amministratore visualizza la \emph{\gls{quotapastog}} dall'app & Soddisfatto \\
\hline
RFO-17    & L'amministratore modifica la \emph{\gls{quotapastog}} dall'app e salva il nuovo valore nel database & Soddisfatto \\
\hline
\end{tabularx}
\end{table}%

\begin{table}%
\caption{Tabella del tracciamento dei requisiti funzionali dal 18 al 35 raggiunti}
%\label{tab:requisiti-funzionali}
\begin{tabularx}{\textwidth}{lXl}
\hline
\textbf{Requisito} & \textbf{Descrizione} & \textbf{Stato}\\
\hline\hline
RFO-18    & L'amministratore visualizza la \emph{\gls{quotastornatag}} degli stagisti dall'app & Soddisfatto \\
\hline
RFO-19    & L'amministratore visualizza la lista con indicato i giorni di presenza degli stagisti dall'app & Soddisfatto \\
\hline
RFO-20    & L'amministratore modifica la lista con indicato i giorni di presenza degli stagisti dall'app e salva le modifiche nel database & Soddisfatto \\
\hline
RFO-21    & L'utente visualizza la lista con indicati i propri giorni di presenza a pranzo dall'app & Soddisfatto \\
\hline
RFO-22    & L'utente modifica la lista con indicati i propri giorni di presenza a pranzo dall'app e salva le modifiche nel database & Soddisfatto \\
\hline
RFO-23    & L'utente si disconnette dall'app & Soddisfatto \\
\hline
RFO-24    & L'utente visualizza i propri dati dall'app & Soddisfatto \\
\hline
RFO-25    & L'utente visualizza la propria mail dall'app & Soddisfatto \\
\hline
RFO-26    & L'utente visualizza il proprio nome dall'app & Soddisfatto \\
\hline
RFO-27    & L'utente visualizza il proprio cognome dall'app & Soddisfatto \\
\hline
RFO-28    & L'utente visualizza il proprio UID Satispay dall'app & Soddisfatto \\
\hline
RFO-29    & L'utente modifica i propri dati dall'app e salva le modifiche nel database & Soddisfatto \\
\hline
RFO-30    & L'utente modifica la propria password dall'app e salva la nuova password nel database & Soddisfatto \\
\hline
RFO-31    & L'utente modifica il proprio nome dall'app e salva il nuovo nome nel database & Soddisfatto \\
\hline
RFO-32    & L'utente modifica il proprio cognome dall'app e salva il nuovo cognome nel database & Soddisfatto \\
\hline
RFO-33    & L'utente modifica il proprio UID Satispay e salva il nuovo UID nel database & Soddisfatto \\
\hline
RFD-34    & Viene chiesto a ChatGPT una possibile ricetta da proporre a pranzo & Non soddisfatto \\
\hline
RFD-35    & Si aggiunge la ricetta proposta da ChatGPT nel menu e si salva la ricetta nel database & Non soddisfatto \\
\hline
\end{tabularx}
\end{table}%

\newpage

\begin{table}[htb]%
\caption{Tabella del tracciamento dei requisiti qualitativi raggiunti}
\label{tab:reqqualitativi-raggiunti}
\begin{tabularx}{\textwidth}{lXl}
\hline
\textbf{Requisito} & \textbf{Descrizione} & \textbf{Stato}\\
\hline\hline
RQN-1    & Il codice \emph{front-end} deve essere coperto da test di unità & Non soddisfatto \\
\hline
\end{tabularx}
\end{table}%

\begin{table}[htb]%
\caption{Tabella del tracciamento dei requisiti di vincolo raggiunti}
\label{tab:reqvincolo-raggiunti}
\begin{tabularx}{\textwidth}{lXl}
\hline
\textbf{Requisito} & \textbf{Descrizione} & \textbf{Stato}\\
\hline\hline
RVO-1    & L'applicazione deve essere sviluppato con il \emph{framework} Flutter & Soddisfatto \\
\hline
RVO-2    & L'applicazione deve essere sviluppato con la piattaforma Firebase & Soddisfatto \\
\hline
RVO-3    & L'applicazione deve essere accessibile su cellulari con sistema operativo Android e iOS & Soddisfatto \\
\hline
RVO-4    & La mail che l'utente deve utilizzare per registrarsi nel database e accedere all'app deve essere fornita da RiskAPP & Soddisfatto \\
\hline
RVO-5    & La mail dell'utente non deve essere modificabile tramite app & Soddisfatto \\
\hline
\end{tabularx}
\end{table}%

