\chapter{Processi e metodologie}
\label{cap:processi-metodologie}

\intro{In questo capitolo viene spiegato il Material Design che sta alla base della progettazione dell'app, viene poi riportato il metodo di lavoro utilizzato e infine le tecnologie adottate per lo sviluppo del progetto.}\\

\section{Material Design}
% Nome, riscrittura senza ripetizioni del contesto (presentazione simil "marketing")
%L'applicazione è stata creata seguendo la regole di Material Design di Goggle  e i consigli di Material Components messi a disposizione da Flutter (https://docs.flutter.dev/ui/widgets/material).\newline
%Per permettere la navigazione tra le varie pagine dellàapplicazione, è stato utilizzato una barra di navigazione posta sulla lato inferiore della schermata (Figura ???)
Alla base dell'applicazione, è stato scelto di seguire il Material Design (Figura \ref{fig:material}) sviluppato da Google, che si concentra su un maggiore uso di \emph{layout} basati su una griglia, animazioni, transizioni ed effetti di profondità come l'illuminazione e le ombre.\newline
Si tratta di una serie di regole ideate per consentire una buona \emph{\gls{ux}}\glsfirstoccur e definire una \emph{\gls{ui}}\glsfirstoccur per l'utente da implementare in ambiente Web, Android e in \emph{\gls{flutterg}}.\newline
Viene annunciato per la prima volta da Google il 25 giugno del 2014 durante il Google I/O, una conferenza organizzata annualmente da Google a Mountain View, in California.\newline
\begin{figure}[!h] 
    \centering 
    \includegraphics[width=0.2\columnwidth]{MaterialDesign} 
    \caption{Logo del Material Design di Google}\label{fig:material}
\end{figure}
\newline
Venne rinnovato nel 2018 con il Material Design 2, anche chiamato Google Material Theme, introducendo un maggiore utilizzo di angoli arrotondati, spazi bianchi e icone colorate, infine viene rinnovato nel 2021 con il Material Design 3, oppure Material You, introducendo l'uso di tasti più grandi e maggiore uso delle animazioni.\newline
Oggi viene ancora utilizzato il Material Design 3 ed è stato seguito per lo sviluppo dell'app dei pranzi.\newline
\newline
Per consentire l'uso dei propri prodotti software a più utenti possibili, il Material Design segue le regole del \emph{\gls{wcag}}\glsfirstoccur, mettendo alla base di ogni progetto l'accessibilità, creando così dei prodotti inclusivi, cioè usabili da tutti i tipi di utenti, anche con disabilità, consentendo a ciascuno un'esperienza fluida e semplice da usare.\newline
\newline
I \emph{layout} devono essere studiati in modo da guidare l'utente nella navigazione della pagina e devono essere dinamici, in modo che le pagine si adattino ad ogni tipo di schermo.\newline
Vengono indicate delle regole precise su come devono essere impostate le \emph{\gls{componentig}}\glsfirstoccur, come devono essere raggruppate, lo spazio che deve esserci e tanti altri piccoli ma importanti dettagli che lo sviluppatore deve considerare per permettere all'utente di orientarsi su qualsiasi dispositivo.\newline
Anche \emph{\gls{flutterg}} offre una guida sulle \emph{\gls{componentig}} che mette a disposizione per lo sviluppatore e che sono state ideate per rispettare le regole di Material Design appena descritte.\newline

\section{Metodo di lavoro}
Durante lo stage, RiskApp contava circa dieci dipendenti e ognuno era incaricato di sviluppare e mantenere una parte della loro piattaforma, confrontandosi tra loro ogni giorno per capire come continuare a lavorare.\newline
Il loro metodo di lavoro si avvicina a un metodo Agile, più precisamente ad uno SCRUM, utilizzato anche per lo sviluppo del progetto di stage.\newline
\newline
Il Manifesto per lo sviluppo Agile (\cite{site:manifesto-agile}) è composto da dodici principi fondamentali che descrivono il modo in cui deve lavorare il team, permettendo possibili cambiamenti in corso d'opera e mettendo al primo posto il cliente, rilasciando varie versioni del prodotto funzionante dopo brevi periodi e privilegiando le comunicazioni faccia a faccia.\newline
Lo SCRUM è un \emph{framework} di gestione dei progetti Agile che mira a cinque valori fondamentali: impegno, focus, apertura, rispetto e coraggio.\newline
Questo \emph{framework} ha acquisito negli ultimi anni una straordinaria popolarità nel mondo dell’informatica grazie ai vantaggi offerti, come maggiore collaborazione con l’utente finale, il suo contributo al miglioramento continuo e la superiore gestione dei rischi.\newline
L’idea di fondo consiste nel suddividere i periodi di lavoro in \emph{sprint} di durata fissata, caratterizzati da un insieme di obiettivi da realizzare (\emph{sprint backlog}).\newline
\newline
Per lo sviluppo del progetto di stage, ogni giorno veniva riportato quanto era stato fatto e veniva mostrato il funzionamento, raccogliendo possibili idee per migliorare o modificare l'app.\newline
Se in corso d'opera venivano incontrate eventuali problematiche sullo sviluppo, si ragionava su come affrontare o modificare il prodotto per risolvere questi problemi, permettendo così di soddisfare ogni esigenza degli utenti finali, in questo caso per soddisfare le esigenze dei dipendenti dell'azienda.

\newpage

\section{Tecnologie}

\subsection{Flutter}
Flutter (Figura \ref{fig:flutter}) è un progetto open-source di Google il cui vantaggio principale è la generazione di applicazioni multipiattaforma a partire da un unico codice sorgente.\newline
Permette quindi allo sviluppatore di concentrarsi sul prodotto da realizzare senza dover preferire un sistema operativo mobile ad un altro.\newline
Per questo motivo è stato scelto di utilizzare Flutter come \emph{framework} principale, dato che il prodotto finale deve funzionare sia per dispositivi Android sia per dispositivi iOS.\newline
\begin{figure}[!h] 
    \centering
    \includegraphics[width=0.4\columnwidth]{Flutter} 
    \caption{Logo di Flutter}\label{fig:flutter}
\end{figure}

\subsection{Dart}
Il linguaggio sul quale si basa Flutter è Dart (Figura \ref{fig:dart}), nato con l’intento di sostituire JavaScript come protagonista delle applicazioni web.\newline
Tra i suoi pregi si elencano il compilatore JIT, migliore gestione della sicurezza, la velocità e la maggiore scalabilità.\newline
Il paradigma principale è l’orientamento agli oggetti, una sua particolarità è data dalla sua attenzione alla \emph{null safety}, per la quale nessun valore può essere nullo a meno che questa possibilità non sia esplicitamente dichiarata.\newline
\begin{figure}[!h] 
    \centering 
    \includegraphics[width=0.4\columnwidth]{Dart} 
    \caption{Logo di Dart}\label{fig:dart}
\end{figure}

\subsection{Firebase}
Firebase (Figura \ref{fig:dart}) è una piattaforma \emph{open-source} per la creazione di applicazioni per dispositivi mobili e web sviluppata da Google.\newline
Firebase sfrutta l'infrastruttura di Google e il suo cloud per fornire una suite di strumenti per scrivere, analizzare e mantenere applicazioni \emph{cross-platform}.\newline
Infatti offre funzionalità come analisi, database (usando strutture noSQL), messaggistica e segnalazione di arresti anomali per la gestione di applicazioni web, iOS e Android.\newline
Per lo sviluppo dell'app sono stati utilizzati:
\begin{itemize}
    \item Firebase Autentication, per permettere la registrazione e l'autenticazione di un utente tramite mail e password;
    \item Cloud Firestore, per la gestione del database.
\end{itemize}
\begin{figure}[!h] 
    \centering 
    \includegraphics[width=0.5\columnwidth]{Firebase} 
    \caption{Logo di Firebase}\label{fig:figma}
\end{figure}

\subsection{Figma}
Figma (Figura \ref{fig:figma}) è un software per la progettazione di User Interface(UI).\newline
Permette infatti di realizzare prototipi delle interfacce, detti anche \emph{mockup}, che permettono di illustrare il risultato finale che si desidera ottenere.\newline
Questo strumento è stato utilizzato per mostrare e concordare l'interfaccia dell'app al tutor aziendale, prima della fase di codifica.\newline
\begin{figure}[!h] 
    \centering 
    \includegraphics[width=0.4\columnwidth]{Figma} 
    \caption{Logo di Figma}\label{fig:figma}
\end{figure}

\subsection{Android Studio}
Android Studio (Figura \ref{fig:android}) è un \emph{\gls{ide}}\glsfirstoccur adibito per la creazione di applicazioni Android e mette a disposizione dei simulatori virtuali di uno o più cellulari con il sistema operativo di Google.\newline
Il progetto è stato sviluppato interamente con l'uso di questo \emph{\gls{ideg}} ed è stato utilizzato il simulatore virtuale di Google Pixel 7 con sistema operativo Android 13 per testare la \emph{\gls{buildg}}\glsfirstoccur dell'app.\newline
\begin{figure}[!h] 
    \centering 
    \includegraphics[width=0.5\columnwidth]{AndroidStudio} 
    \caption{Logo di Android Studio}\label{fig:android}
\end{figure}

\subsection{Xcode}
Xcode (Figura \ref{fig:apple}) è un \emph{\gls{ideg}} completamente sviluppato e mantenuto da Apple, contenente una suite di strumenti utili allo sviluppo di software per i sistemi macOS, iOS, iPadOS, watchOS e tvOS.\newline
Per poter testare la \emph{\gls{buildg}} del progetto, è stato utilizzato il simulatore virtuale di iPhone 15 con sistema operativo iOS 17, messo a disposizione da questo software.\newline
\begin{figure}[!h] 
    \centering 
    \includegraphics[width=0.25\columnwidth]{Xcode} 
    \caption{Logo di Xcode}\label{fig:apple}
\end{figure}

\subsection{GitHub}
GitHub (Figura \ref{fig:github}) è una piattaforma di \emph{hosting} per per ospitare codice all'interno di repository basato sul software Git.\newline
Fornisce agli sviluppatori strumenti per migliorare e mantenere il codice come:
\begin{itemize}
    \item \emph{features} utilizzabili da linea di comando;
    \item gestione delle \emph{pull request} e \emph{code review};
    \item strumenti per l’\emph{issue tracking}.
\end{itemize}
La codebase della piattaforma RiskApp è suddivisa in varie repository su GitHub.\newline
Per questo progetto, l'azienda ha riservato una repository apposita per permettermi di lavorare in autonomia al codice.\newline
\begin{figure}[!h] 
    \centering 
    \includegraphics[width=0.5\columnwidth]{GitHub} 
    \caption{Logo di GitHub}\label{fig:github}
\end{figure}

\newpage

\subsection{Slack}
Slack (Figura \ref{fig:slack}) è un applicazione multipiattaforma per la messaggistica istantanea tra membri di un gruppo di lavoro.\newline
Una delle funzioni di Slack è la possibilità di organizzare la comunicazione del team attraverso canali specifici, canali che possono essere accessibili a tutto il team o solo ad alcuni membri.\newline
È possibile inoltre comunicare con il team anche attraverso chat individuali private o chat con due o più membri.\newline
Questo software è stato utilizzato per comunicare con il tutor aziendale da remoto e per condividere materiale.\newline
\begin{figure}[!h] 
    \centering 
    \includegraphics[width=0.4\columnwidth]{Slack} 
    \caption{Logo di Slack}\label{fig:slack}
\end{figure}

